%-----------------------------------------------
% Assignment 3: Fundamental Image Processing Techniques Using Python
%-----------------------------------------------
\refstepcounter{section}
% \addlabcontentsline{Assignment \thesection: Fundamental Image Processing Techniques}{28-07-2025}{\thepage}
\addlabcontentsline{Assignment~\thesection: Fundamental Image Processing Techniques}{15-09-2025}{\thepage}

\section*{\centering Assignment \thesection: Fundamental Image Processing Techniques}

\noindent \textbf{Objective:} To apply fundamental image processing techniques using Python. This assignment focuses on
real-world mini-tasks including image manipulation, sampling, color conversion, and
neighborhood filtering. The goal is to develop a solid foundation in working with digital images
programmatically.
\vspace{-1em}


%-----------------------------------------------
% Task 1
%-----------------------------------------------
\subsection{Task 1:  Create an ID Card Image with Name and Roll Number Overlay}


\vspace{-0.1em}
\subsubsection*{Code Implementation}
% \vspace{-1em}
\begin{lstlisting}[style=pythonStyle, caption={}]
import cv2

# Load image
img = cv2.imread("/home/yugal/Desktop/Python Programs IP LAB/LAB 3/sample_photo.jpg")

# Resize to 150x150
img = cv2.resize(img, (150, 150))

# Overlay text (keep it readable and centered)
cv2.putText(img, "Name: Yugal", (20, 80), cv2.FONT_HERSHEY_SIMPLEX, 0.4, (0, 0, 0), 1, cv2.LINE_AA)
cv2.putText(img, "Roll: 25903053", (20, 100), cv2.FONT_HERSHEY_SIMPLEX, 0.4, (0, 0, 0), 1, cv2.LINE_AA)

# Add white border
id_card = cv2.copyMakeBorder(img, 10, 10, 10, 10, cv2.BORDER_CONSTANT, value=[255, 255, 255])

# Save output
cv2.imwrite("id_card_output.jpg", id_card)

print("ID Card saved as id_card_output.jpg")
    
\end{lstlisting}

\vspace{1em} 
% \subsubsection*{Output}
\begin{outputbox}
\begin{verbatim}

ID Card saved as id_card_output.jpg

\end{verbatim}
\end{outputbox}

\vspace{-1.5em}
\subsubsection*{Output}
\begin{figure}[H]
    \centering
    \vspace{-3em}
    \includegraphics[width=0.3\textwidth]{Lab3/id_card_output.jpg}
    \caption{ID Card Output}
    \label{fig:id_card_output}
    \vspace{-1em}
\end{figure}

%-----------------------------------------------
% Task 2
%-----------------------------------------------
\subsection{Task 2: Convert a color image to Grayscale, HSV, and LAB and compare for edge detection.
}


\vspace{-0.1em}
\subsubsection*{Code Implementation}
% \vspace{-1em}
\begin{lstlisting}[style=pythonStyle, caption={}]
import cv2
import matplotlib.pyplot as plt

img = cv2.imread("/home/yugal/Desktop/Python Programs IP LAB/LAB 3/task_2.jpg")
img_rgb = cv2.cvtColor(img, cv2.COLOR_BGR2RGB)   # convert BGR -> RGB for matplotlib

gray_image = cv2.cvtColor(img, cv2.COLOR_BGR2GRAY)
hsv_image = cv2.cvtColor(img, cv2.COLOR_BGR2HSV)
lab_image = cv2.cvtColor(img, cv2.COLOR_BGR2LAB)

# Plot all versions
plt.figure(figsize=(10,8))

plt.subplot(2,2,1)
plt.imshow(img_rgb)
plt.title("Original Image")
# plt.savefig("Original_Image.png")
plt.axis("off")

plt.subplot(2,2,2)
plt.imshow(gray_image, cmap="gray")
plt.title("Grayscale")
# plt.savefig("Grayscale_Image.png")
plt.axis("off")

plt.subplot(2,2,3)
plt.imshow(cv2.cvtColor(hsv_image, cv2.COLOR_HSV2RGB))  # fix colors for matplotlib
plt.title("HSV")
# plt.savefig("HSV_Image.png")
plt.axis("off")

plt.subplot(2,2,4)
plt.imshow(cv2.cvtColor(lab_image, cv2.COLOR_LAB2RGB))  # fix colors for matplotlib
plt.title("LAB")
# plt.savefig("LAB_Image.png")
plt.axis("off")

plt.suptitle("Grayscale is best for edge detection; HSV is effective for segmentation.", fontsize=12)
plt.savefig("Output.png")
plt.show()
    
\end{lstlisting}

% \vspace{1em} 
% \subsubsection*{Output}
\vspace{-1.5em}
\subsubsection*{Output}
\begin{figure}[H]
    \centering
    \vspace{-2em}
    \includegraphics[width=1.0\textwidth]{Lab3/Output.png}
    \caption{Grayscale, HSV, and LAB Comparison}
    \label{fig:grayscale_hsv_lab_comparison}
    \vspace{-1em}
\end{figure}

%-----------------------------------------------
% Task 3
%-----------------------------------------------
\subsection{Task 3: Perform resizing, rotation, and vertical flip on an image and save results.}


\vspace{-0.1em}
\subsubsection*{Code Implementation}
% \vspace{-1em}
\begin{lstlisting}[style=pythonStyle, caption={Resize to 50 percentage and 200 percentage of original size}]
import cv2

import matplotlib.pyplot as plt

# Load image
img = cv2.imread("/home/yugal/Desktop/Python Programs IP LAB/LAB 3/task_3.jpg")
img_rgb = cv2.cvtColor(img, cv2.COLOR_BGR2RGB)

# Resize images
resized_50 = cv2.resize(img, None, fx=0.5, fy=0.5)
resized_50_rgb = cv2.cvtColor(resized_50, cv2.COLOR_BGR2RGB)

resized_200 = cv2.resize(img, None, fx=2.0, fy=2.0)
resized_200_rgb = cv2.cvtColor(resized_200, cv2.COLOR_BGR2RGB)

# Helper function to show and save figure
def show_and_save(image, title, filename):
    h, w = image.shape[:2]
    plt.figure(figsize=(w/100, h/100))  # keep actual size
    plt.imshow(image)
    plt.title(f"{title} ({w}x{h})")
    plt.axis("off")
    plt.savefig(filename, dpi=100, bbox_inches="tight")
    plt.show()

# Display and save separately

show_and_save(img_rgb, "Original Image", "original_image.png")

show_and_save(resized_50_rgb, "50% Resize", "resize_50_image.png")

show_and_save(resized_200_rgb, "200% Resize", "resize_200_image.png")
    
\end{lstlisting}

\vspace{1em} 
% \subsubsection*{Output}
\vspace{-1.5em}
\subsubsection*{Output}
\begin{figure}[H]
    \centering
    \vspace{-2em}
    \includegraphics[width=0.6\textwidth]{Lab3/Original_Image.png}
    \caption{Original Image}
    \label{fig:original_image_mountain}
    \vspace{-1em}
\end{figure}

\begin{figure}[H]
    \centering
    % \vspace{-2em}
    \includegraphics[width=0.4\textwidth]{Lab3/resize_50_image.png}
    \caption{50 percentage Resize}
    \label{fig:resize_50_image}
    \vspace{-1em}
\end{figure}

\begin{figure}[H]
    \centering
    \vspace{-2em}
    \includegraphics[width=1.0\textwidth]{Lab3/resize_200_image.png}
    \caption{200 percentage Resize}
    \label{fig:resize_200_image}
    \vspace{-1em}
\end{figure}

\begin{lstlisting}[style=pythonStyle, caption={Rotate by 45 degrees and Flip vertically}]
import cv2

img = cv2.imread("/home/yugal/Desktop/Python Programs IP LAB/LAB 3/task_3.jpg")

(h , w) = img.shape[:2]
center = (w // 2, h//2)

M = cv2.getRotationMatrix2D(center, 45, 1.0)

#warpAffine actually applies the rotation matrix M to the image.
rotated = cv2.warpAffine(img, M, (w,h))

flipped = cv2.flip(img, 0) # 0 means upside down

cv2.imshow("Originial Image ", img)
cv2.imshow("Rotated 45",rotated)
cv2.imshow("Flipped Vertically ",flipped)

cv2.waitKey(0)
cv2.destroyAllWindows()

cv2.imwrite("flipped_image.jpg", flipped)
cv2.imwrite("rotated_image.jpg", rotated)

print("Flipped Image Saved")
print("Vertical Image Saved")
    
\end{lstlisting}

\begin{figure}[H]
    \centering
    % \vspace{-2em}
    \includegraphics[width=0.4\textwidth]{Lab3/flipped_image.jpg}
    \caption{Flipped Image}
    \label{fig:flipped_image}
    \vspace{-1em}
\end{figure}

\begin{figure}[H]
    \centering
    % \vspace{-2em}
    \includegraphics[width=0.4\textwidth]{Lab3/rotated_image.jpg}
    \caption{Rotated Image}
    \label{fig:rotated_image}
    \vspace{-1em}
\end{figure}

%-----------------------------------------------
% Task 4
%-----------------------------------------------
\subsection{Task 4: Downsample and upsample an image using different interpolation methods and display comparisons.}


\vspace{-0.1em}
\subsubsection*{Code Implementation}
% \vspace{-1em}
\begin{lstlisting}[style=pythonStyle, caption={}]
import cv2
import matplotlib.pyplot as plt

# Load image
img = cv2.imread("/home/yugal/Desktop/Python Programs IP LAB/LAB 3/task_4.jpg")
img = cv2.cvtColor(img, cv2.COLOR_BGR2RGB)
h, w, c = img.shape

# Downsample
down2 = img[::2, ::2]
down4 = img[::4, ::4]
down8 = img[::8, ::8]

# Upsample back
def upsample(img_small):
    return [
        cv2.resize(img_small, (w,h), interpolation=cv2.INTER_NEAREST),
        cv2.resize(img_small, (w,h), interpolation=cv2.INTER_LINEAR),
        cv2.resize(img_small, (w,h), interpolation=cv2.INTER_CUBIC)
    ]

up2_nearest, up2_linear, up2_cubic = upsample(down2)
up4_nearest, up4_linear, up4_cubic = upsample(down4)
up8_nearest, up8_linear, up8_cubic = upsample(down8)

# Arrange in 3x3 grid
all_images = [
    up2_nearest, up2_linear, up2_cubic,
    up4_nearest, up4_linear, up4_cubic,
    up8_nearest, up8_linear, up8_cubic
]

titles = [
    "Down2 + Nearest", "Down2 + Linear", "Down2 + Cubic",
    "Down4 + Nearest", "Down4 + Linear", "Down4 + Cubic",
    "Down8 + Nearest", "Down8 + Linear", "Down8 + Cubic"
]

plt.figure(figsize=(12,12))

for i, (im, t) in enumerate(zip(all_images, titles), 1):
    plt.subplot(3,3,i)
    plt.imshow(im)
    plt.title(t, fontsize=10)
    plt.axis("off")

plt.tight_layout()

plt.savefig("upsampled_grid.png", dpi=150, bbox_inches="tight")

plt.show()

# Save original separately
plt.figure(figsize=(6,6))

plt.imshow(img)

plt.title("Original Image")

plt.axis("off")

plt.savefig("original_image.png", dpi=150, bbox_inches="tight")

plt.show()
    
\end{lstlisting}

\vspace{-1em} 
\subsubsection*{Output}

\begin{figure}[H]
    \centering
    % \vspace{-2em}
    \includegraphics[width=1.0\textwidth]{Lab3/upsampled_grid.png}
    \caption{Upsampled Images}
    \label{fig:upsampled_images}
    \vspace{-1em}
\end{figure}


\begin{figure}[H]
    \centering
    % \vspace{-2em}
    \includegraphics[width=0.5\textwidth]{Lab3/original_image copy.png}
    \caption{Original Image}
    \label{fig:original_image}
    \vspace{-1em}
\end{figure}



%-----------------------------------------------
% Task 5
%-----------------------------------------------
\subsection{Task 5: Grayscale image - quantize to 4-bit and 2-bit - show versions and compare quality.}


\vspace{-0.1em}
\subsubsection*{Code Implementation}
% \vspace{-1em}
\begin{lstlisting}[style=pythonStyle, caption={}]
import cv2
import numpy as np
import matplotlib.pyplot as plt
import os

# Check file
print(os.path.exists("/home/yugal/Desktop/Python Programs IP LAB/LAB 3/task_5.jpg"))

# 1. Load image
img = cv2.imread("/home/yugal/Desktop/Python Programs IP LAB/LAB 3/task_5.jpg")
gray = cv2.cvtColor(img, cv2.COLOR_BGR2GRAY)

# 2. Quantize
# 4-bit (16 levels)
step4 = 16
quant4 = (gray // step4) * step4

# 2-bit (4 levels)
step2 = 64
quant2 = (gray // step2) * step2

# 3. Compute MSE
mse_4bit = np.mean((gray - quant4) ** 2)
mse_2bit = np.mean((gray - quant2) ** 2)

# 4. Display and save
plt.figure(figsize=(12,4))

plt.subplot(1,3,1)
plt.imshow(gray, cmap='gray')
plt.title("Original (8-bit)")
plt.axis("off")

plt.subplot(1,3,2)
plt.imshow(quant4, cmap='gray')
plt.title(f"4-bit (16 levels)\nMSE={round(mse_4bit,1)}\nQuality: Good")
plt.axis("off")

plt.subplot(1,3,3)
plt.imshow(quant2, cmap='gray')
plt.title(f"2-bit (4 levels)\nMSE={round(mse_2bit,1)}\nQuality: Poor")
plt.axis("off")

# Save the whole figure
plt.tight_layout()
plt.savefig("quantization_comparison.png", dpi=150, bbox_inches="tight")
plt.show()
    
\end{lstlisting}

\vspace{1em} 
% \subsubsection*{Output}

\begin{figure}[H]
    \centering
    % \vspace{-2em}
    \includegraphics[width=1\textwidth]{Lab3/quantization_comparison.png}
    \caption{Quantization Comparison}
    \label{fig:quantization_comparison}
    \vspace{-1em}
\end{figure}

%-----------------------------------------------
% Task 6
%-----------------------------------------------
\subsection{Task 6: Convert image to grayscale, extract 3 by 3 neighborhood, apply mean sobel and sharpening filters and display results.}


\vspace{-0.1em}
\subsubsection*{Code Implementation}
\vspace{-0.1em}
\begin{lstlisting}[style=pythonStyle, caption={}]
import cv2
import numpy as np
import matplotlib.pyplot as plt

# ------------------------------
# STEP 1: Load image and convert to grayscale
# ------------------------------
img = cv2.imread("/home/yugal/Desktop/Python Programs IP LAB/LAB 3/task_5.jpg")

gray = cv2.cvtColor(img, cv2.COLOR_BGR2GRAY)

# ------------------------------
# STEP 2: Extract 3x3 neighborhood at (100,100)
# ------------------------------
row, col = 100, 100

block = gray[row-1:row+2, col-1:col+2]

print("3x3 neighborhood at (100,100):")

print(block)

# ------------------------------
# STEP 3: Apply filters
# ------------------------------

# (a) Mean Filter
mean_kernel = np.ones((3,3), np.float32) / 9

mean_img = cv2.filter2D(gray, -1, mean_kernel)

# (b) Sobel Edge Detection
sobelx = cv2.Sobel(gray, cv2.CV_64F, 1, 0, ksize=3)
sobely = cv2.Sobel(gray, cv2.CV_64F, 0, 1, ksize=3)
sobel_img = cv2.magnitude(sobelx, sobely)
sobel_img = np.uint8(np.clip(sobel_img, 0, 255))  # convert to uint8

# (c) Sharpening
sharpen_kernel = np.array([[0,-1,0],
                           [-1,5,-1],
                           [0,-1,0]])

sharp_img = cv2.filter2D(gray, -1, sharpen_kernel)

# ------------------------------
# STEP 4: Save individual filtered results
# ------------------------------

def save_filtered(image, title, comment, filename):
    plt.figure(figsize=(8,8))
    plt.imshow(image, cmap='gray')
    plt.axis('off')
    plt.title(title)
    plt.figtext(0.5, 0.2, comment, ha='center', fontsize=12)
    plt.savefig(filename, dpi=150, bbox_inches='tight')
    plt.close()  # close to avoid overlapping figures

# Save Mean Filter
save_filtered(mean_img, "Mean Filter (Blurry)",
              "Comment: Mean filter smooths the image but edges are blurred.",
              "mean_filter.png")

# Save Sobel Edge Detection
save_filtered(sobel_img, "Sobel Edges",
              "Comment: Sobel highlights edges clearly; good for edge clarity.",
              "sobel_edges.png")

# Save Sharpened
save_filtered(sharp_img, "Sharpened",
              "Comment: Sharpened image enhances contrast and makes edges more prominent.",
              "sharpened.png")

print("All filtered images saved successfully.")
   
\end{lstlisting}

\vspace{1em} 
% \subsubsection*{Output}
\begin{outputbox}
3x3 neighborhood at (100,100):

[[17 12 12]

[17  6 11]

[16  9  6]]

All filtered images saved successfully.
\end{outputbox}

\begin{figure}[H]
    \centering
    % \vspace{-2em}
    \includegraphics[width=0.7\textwidth]{Lab3/mean_filter.png}
    \caption{Mean Filter}
    \label{fig:mean_filter}
    \vspace{-1em}
\end{figure}


\begin{figure}[H]
    \centering
    % \vspace{-2em}
    \includegraphics[width=0.8\textwidth]{Lab3/sobel_edges.png}
    \caption{Sobel Edges}
    \label{fig:sobel_edges}
    \vspace{-1em}
\end{figure}

\begin{figure}[H]
    \centering
    % \vspace{-2em}
    \includegraphics[width=1\textwidth]{Lab3/sharpened.png}
    \caption{Sharpened}
    \label{fig:sharpened}
    \vspace{-1em}
\end{figure}


%-----------------------------------------------
% End of Assignment 3
%-----------------------------------------------

